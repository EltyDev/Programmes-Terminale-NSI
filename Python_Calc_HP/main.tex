\documentclass[a4paper]{article}
\usepackage[T1]{fontenc}
\usepackage[utf8]{inputenc}
\usepackage{lmodern}
\usepackage[english]{babel}
\usepackage{amsthm}
\usepackage{amsfonts}
\usepackage{amsmath}
\begin{document}
\section*{La calculatrice HP}
\section{Petit historique de HP}
\subsection{La gènése}
William Hewlett et David Packard fondent leur société en 1939. L'entreprise est installée en Californie, c'est la création de la emph{Silicon Valley}.

L'ordre des noms, Hewlett puis Packard à été déterminée au hasard, mais D. Packard avait gagné mais a laissé sa place à son ami. \`A l'origine, Hewlett est l'ingénieur et Packard est le gestionnaire.

\subsection{Premier produits}
Le premise produit d'HP devait être un générateur de fréquence audio; le HP 200A. Il coutait moins cher que ses concurents et était performant. Il fût vendu en 8 exemplaires à \emph{Disney} pour Fantasia.

Dans les années suivantes les machines HP se diversifièrent, et cherchèrent à construire de appareils de plus en plus performants.

\subsection{L'informatique}
En 1956, HP fonda Dynac, le logo étant symplement retourné, qui se spécialise dans la fabrications d'équipement informatique. La marque fût intégrée à la maison mère en 1959.

En 1966, HP se lenca dans la création de mini-ordinateur, avec le HP-1000. C'est le début des "calculateur personnels" HP. 
\section{Petit historique de la notation RPN}
Les calculatrice HP utilisent la notation polosaise inversée \emph{Reverse Polish Notation}.

\subsection{La logique RPN}
Cette notation à été inventée en 1920, pour couder des expressions mathématiques sans parenthèses, mais quand même capable de traiter toutes formules. Dans cette solutions, les opérateurs sont avant les argument.

Ces notations sont utilisées en premier dans les micro-ordinateur, car c'est plus simple pour les processeurs de l'époque.

Cette notation utilise la pile opérationnelle.

\section{Le fonctionnement de la calculatrice}

Il faut rentrer les valeurs, puis faire les opérations.
\smallbreak
$2$ $2$ $+$ rendra $4$
\smallbreak
$2$ $3$ $*$ rendra $6$
\smallbreak
Et tout est enregistrer dans une pile, mais lorsque des valeurs sont utilisées, elle sont remplacées par leur résultats dans la pile.
\smallbreak
Si la pile est
\smallbreak
$1$
\smallbreak
$2$
\smallbreak
$3$
\smallbreak
et que l'on fait $+$, la pile sera : 
\smallbreak
$3$
\smallbreak
$3$
\smallbreak
Les opérations disponibles sont +,-, $*$ $(\times)$, /, ** (puissance ), et les fonctions qui fonctionnent qu'avec le dernier chiffre : sin, cos, tan, exp, log, log10 













\end{document}